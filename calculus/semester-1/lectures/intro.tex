\lesson{1}{13 авг 2020}{Введение}

\part{Предел}

\nchapter{2}{Representations of finite groups}

\section{Предел функции}

\begin{definition}
  Функция $f:N \to X$, областью определения которой является множество натуральных чисел,
  называется \it{последовательностью}
\end{definition}

Запишем теперь приведенные формулировки определения предела в логической символике, договорившись,
что запись << $\lim_{n \to \infty}{x_{n}} = A $ >>, означает, что $A$ - предел последовательности $\{ x_{n} \}$. Итак,

\[
  \boxed{
    ( \lim_{n \to \infty}{x_{n}} = A ) := \forall V(A) \ \exists N \in \N \ \forall n > N \ ( x_{n} \in V(A) )
 }
\]

и соответсвенно

\[
  ( \lim_{n \to \infty}{x_{n}} = A ) := \forall \epsilon > 0 \ \exists N \in \N \ \forall n > N \ ( | x_{n} - A | < \epsilon )
\]

\begin{theorem}
  Финально постоянная последовательность сходится
\end{theorem}

\begin{proof}
  Если $x_{n} = A$ при $n>N$, то для любой окрестности $V(A)$ точки $A$ имеем $x_{n} \in V(A)$ при $n>N$,
  т.е. $ \lim_{n \to \infty}{x_{n}} = A$
\end{proof}

\begin{notation}
   Запись << $\lim_{n \to \infty}{x_{n}} = A $ >>, означает, что
   $A$ - предел последовательности $\{ x_{n} \}$.
\end{notation}
$\card$
test
